%%%%%%%%%%%%%%%%%%%%%%%%%%%%%%%%%%%%%%
% LaTeX poster template
% Created by Nathaniel Johnston
% August 2009
% http://www.nathanieljohnston.com/2009/08/latex-poster-template/
%%%%%%%%%%%%%%%%%%%%%%%%%%%%%%%%%%%%%%

\documentclass[final]{beamer}
\usepackage[scale=1.24]{beamerposter}
\usepackage{graphicx}           % allows us to import images
\usepackage{sidecap}
\usepackage{wrapfig}
\usepackage{caption}
\usepackage{sidecap}
\usepackage{cleveref} %add option [capitalize]?
\usepackage{subfig}
% \usepackage{natbib}


%-----------------------------------------------------------
% Define the column width and poster size
% To set effective sepwid, onecolwid and twocolwid values, first choose how many columns you want and how much separation you want between columns
% The separation I chose is 0.024 and I want 4 columns
% Then set onecolwid to be (1-(4+1)*0.024)/4 = 0.22
% Set twocolwid to be 2*onecolwid + sepwid = 0.464
%-----------------------------------------------------------

\newlength{\sepwid}
\newlength{\onecolwid}
\newlength{\twocolwid}
\newlength{\threecolwid}
%trying to make this easily changeable. 
%Line 199 of beamerthemeconfposter.sty was hardcoded
%iine 188 originally uses \linewidth but I cant figure out how to change that.
%frame seems to be too wide as well. Somehow my columns are not changeing size?
\setlength{\paperwidth}{48in}%{48in}
\setlength{\paperheight}{36in}%{36in}
\setlength{\sepwid}{0.024\paperwidth}
\setlength{\onecolwid}{0.22\paperwidth}%should be 0.22
\setlength{\twocolwid}{0.464\paperwidth}%should be 0.464
\setlength{\threecolwid}{0.708\paperwidth}
\setlength{\topmargin}{-0.5in}
\usetheme{confposter}
\usepackage{exscale}

%-----------------------------------------------------------
% The next part fixes a problem with figure numbering. Thanks Nishan!
% When including a figure in your poster, be sure that the commands are typed in the following order:
% \begin{figure}
% \includegraphics[...]{...}
% \caption{...}
% \end{figure}
% That is, put the \caption after the \includegraphics
%-----------------------------------------------------------

\usecaptiontemplate{
\small
\structure{\insertcaptionname~\insertcaptionnumber:}
\insertcaption}

%-----------------------------------------------------------
% Define colours (see beamerthemeconfposter.sty to change these colour definitions)
%-----------------------------------------------------------

\setbeamercolor{block title}{fg=dblue,bg=white}
\setbeamercolor{block body}{fg=black,bg=white}
\setbeamercolor{block alerted title}{fg=white,bg=dblue!70}
\setbeamercolor{block alerted body}{fg=black,bg=dblue!10}

\setbeamercolor{item}{fg=dblue}
\setbeamercolor{item projected}{fg=white,bg=dblue}

%-----------------------------------------------------------
% Name and authors of poster/paper/research
%-----------------------------------------------------------

\title{Correlating Type Ia Supernova Properties with Their Local Environment \\Using HST Snapshots of Host Galaxies} %todo:make this shorter.
%from abstract: Correlating Type Ia Supernova Properties with Their Local Environment Using HST Snapshots of Host Galaxies
%i like: Correlating Type Ia supernova properties with their local environment using HST Snapshots of host galaxies
\author{Benjamin Rose \& Peter Garnavich}
\institute{Department of Physics, University of Notre Dame, Notre Dame, IN 46556\\ \vskip1ex brose3@nd.edu}

%-----------------------------------------------------------
% Start the poster itself
%-----------------------------------------------------------

\begin{document}
\begin{frame}[t]
\begin{columns}[t]

\begin{column}{\sepwid}\end{column}

\begin{column}{\onecolwid} \label{col1}
\begin{alertblock}{The big question}
% What information about Type Ia supernovae will allow us to reduce the scatter on the Hubble diagram?\\
% For Type Ia supernovae to be standard candles their variabilities need to be understood and corrected. 
% In order to improve distance measurements, how can Type Ia supernovae variabilities be better understood and the scatter on the Hubble diagram reduced?
Type Ia supernovae are excellent distance indicators, but their Hubble residuals show some correlation with host galaxy properties. Can using information on the local environment reduce Type Ia systematic
distance errors?
\end{alertblock}

\vskip1ex

%todo:emphaize luminosity-decline rate, color, $x_1$, host galaxy mass, global mentality, and local H$_\alpha$
\begin{block}{Basic SNIa standardization}
Type Ia supernovae (SNIa) are used as a standard candle to measure cosmological distances. Each individual SNIa has some variability, but these can be standardized by finding relationships between properties of the SNIa themselves. The two most popular light curve fitters take these relationships into account like luminosity-decline rate \cite{Phillips93}, color, and $x_1$.
\end{block}

\vskip1ex

\begin{block}{Environmental effects}
Research over the past several years is indicating that external variables seem to also contribute to SNIa variability. There are still uncertainties about how the host galaxy and the local environment influence the luminosity, color, and Hubble residuals of SNIa. 
%These global trends have been researched including metallicity \cite{Hayden13}, and host mass \cite{Childress13}. Other?
%2013ge is one of three from Childress that year.
%Local effects have also been lookedAnd something else about local effect, $H_\alpha$ x2 \cite{Rigault13,Rigault14}. 
There has been research that show the effects of host galaxy mass \cite{Childress13}, global metallicity \cite{Hayden13}, and local H$_\alpha$ \cite{Rigault13,Rigault14}.

% \vskip1ex
% \begin{itemize}
%     \item global metallicity \cite{Hayden13}
%     \item host galaxy mass (mass step) \cite{Childress13}
%     \item local H$_\alpha$ x2 \cite{Rigault13,Rigault14}
% \end{itemize}
% \vskip1ex
\end{block}

\vskip1ex

\begin{block}{Data collection}
% Using the Hubble Space Telescope (HST) we 
We investigate these questions by analyzing high angular resolution Hubble Space Telescope (HST) imaging of SDSS-II host galaxies. 
The SDSS-II SN Survey obtained 500 spectroscopically confirmed type SNIa and another 500 SNIa based on light curves. 
The HST data collected are ``snapshot'' images obtained while the telescope was slewing to new targets; so the total exposure times are less than 30 minutes. ACS images were obtained in F475W and F625W filters, similar to SDSS g- and r-bands. In total, we observed 61 host galaxies in Stripe 82 that had SNIa discovered by the SDSS-II SN Survey.
\end{block}
\end{column}

\begin{column}{\sepwid}\end{column}

\begin{column}{\twocolwid}\label{col2}
\begin{block}{HST vs SDSS data}
The HST images are combined F475W and F625W images and then Gaussian smoothed with a 1 pixel standard deviation. The location of the SNIa is denoted with the star. For these images the pixel size is 0.04 arcsec.
%Color SDSS images are RGB of SDSS ??? filters (blame Peter).
Color SDSS images are from SDSS DR12, with the location of the SNIa at the center of the cross-hairs. For these images the pixel size is 0.396 arcsec.
%todo:make stars larger in HST images.
%todo:what is up with is up with SDSS cross-hairs not looking like the same place as HST?
%is WCS shift important? 

%%%%%%%%%%%%%%%%%%%%%%%%%%%%%%%%%%
%try side by side. (4 photos wide)
%%%%%%%%%%%%%%%%%%%%%%%%%%%%%%%%%%
\end{block}

% \vskip2ex

\begin{columns}
\begin{column}{\onecolwid}
\begin{figure}
    \centering
    \subfloat[HST, one pixel is 145 pc.][t]{8in}
        \includegraphics[height=5.2in]{SDSS-SN14984.png}
    }
\caption{Host galaxy of SDSS SN14984 (SN 2006js), at a redshfit of z=0.183.}
\end{figure}

\begin{figure}
    \centering
    \subfloat[HST, one pixel is 54.2 pc.]{
        \includegraphics[height=5.2in]{SN6057.png}
    }
    \subfloat[SDSS, one pixel is 536 pc.]{
        \includegraphics[height=5.2in]{SDSS-SN6057.png}
    }
\caption{Host galaxy of SDSS SN6057 (SN 2005if), at a redshift of z=0.0665.}
\end{figure}

\begin{figure}
    \centering
    \subfloat[HST, one pixel is 114 pc.][t]{8in}
        \includegraphics[height=5.2in]{SDSS-SN2635.png}
    }
\caption{Host galaxy of SDSS SN2635 (2005fw), at a redshfit of z=0.143.}
\end{figure}
\end{column}

\begin{column}{\sepwid}\end{column}

\begin{column}{\onecolwid}
\begin{figure}
    \centering
    \subfloat[HST, one pixel is 36.3 pc.][t]{8in}
        \includegraphics[height=5.2in]{SDSS-SN14279.png}
    }
\caption{Host galaxy of SDSS SN14279 (SN2006ne), at a redshfit of z=0.0443.}
\end{figure}

\begin{figure}
    \centering
    \subfloat[HST, one pixel is 33.3 pc.][t]{8in}
        \includegraphics[height=5.2in]{SDSS-SN17886.png}
    }
\caption{Host galaxy of SDSS SN17886 (SN2007jh), at a redshfit of z=0.0407.}
\end{figure}

\begin{figure}
    \centering
    \subfloat[HST, one pixel is 191 pc.][t]{8in}
        \includegraphics[height=5.2in]{SDSS-SN12874.png}
    }
\caption{Host galaxy of SDSS SN12874 (SN2006fb), at a redshfit of z=0.245.}
\end{figure}
\end{column}
\end{columns}
%%%%%%%%%%%%%%%%%%%%%%%%%%%%%%%%%%
%%%%%%%%%%%%%%%%%%%%%%%%%%%%%%%%%%

%%%%%%%%%%%%%%%%%%%%%%%%%%%%%%%%%%
%Two images wide.
%%%%%%%%%%%%%%%%%%%%%%%%%%%%%%%%%%
% \vskip2ex

% \begin{figure}
% % \includegraphics[scale=1]{SN14984-HST.png}
    
%     \centering
%     \subfloat[HST]{%[t]{8in}
%         \includegraphics[height=5.5in]{SN14984-HST.png}%h=5.5in is about scale=1
%     } \hskip5ex
%     \subfloat[SDSS color image]{%}[t]{8in}
%         \includegraphics[height=5.5in]{SDSS-SN14984-zoomed.pdf}
%     }
% \caption{Host galaxy of SDSS SN14984}
% \end{figure}

% \vskip2ex

% \begin{figure}
% % \includegraphics[scale=1]{SN14984-HST.png}
    
%     \centering
%     \subfloat[HST]{%[t]{8in}
%         \includegraphics[scale=1]{SN14984-HST.png}
%     } \hskip5ex
%     \subfloat[Should not be HST]{%}[t]{8in}
%         \includegraphics[scale=1]{SN14984-HST.png}
%     }
% \caption{THis should not be the host galaxy of SDSS SN14984}
% \end{figure}

% \end{block}
%%%%%%%%%%%%%%%%%%%%%%%%%%%%%%%%%%
%%%%%%%%%%%%%%%%%%%%%%%%%%%%%%%%%%
\end{column}

\begin{column}{\sepwid}\end{column}

\begin{column}{\onecolwid}\label{col3}
\begin{alertblock}{Why HST?}
% The HST images show both more large scale structure (like SN14984) and local ``clumps'' and other small features (like SN12874).
With HST's higher angular resolution a SNIa's ``local'' environment is more precisely defined, allowing for more accurate environmental classification in galaxies with detailed structure. For a typical galaxy at z=0.1, the projected resolution element goes from $\sim 3~\text{kpc}$ for SDSS to $\sim 160~\text{pc}$.% for HST.
\end{alertblock}

\vskip1ex

\begin{block}{What is next}
HST's resolution and low background allow for detailed analysis of both the region around the SNIa and the galaxy as a whole. Co-added SDSS-II images of the hosts are used to supplement the HST data in regions of low surface brightness. From this data set we will estimate the fractional pixel rank and photometric color of the SNIa's location, correlating the local environment variables with SNIa luminosity, light curve width, color, and Hubble residual. Then we will assess the impact of these correlations on the accuracy of SNIa distance estimates and possible biases in measuring the Hubble constant and dark energy parameters.
\end{block}

\vskip1ex

\begin{block}{References} 
% \bibliographystyle{apj}
%using small, but could also use footnotesize, but other sizes are too small.
\small{\begin{thebibliography}{99}
\bibitem{Childress13}Childress, Aldering, Antilogus, et al. 2013, ApJ, 770, 108
%I may want another 
\bibitem{Hayden13}Hayden, Gupta, Garnavich, et al. 2013, ApJ, 764, 191
\bibitem{Phillips93}Phillips, M. M. 1993, ApJ, L105, 413
\bibitem{Rigault13} Rigault, Copin, Aldering, et al. 2013, A\&A, 560, A66
\bibitem{Rigault14} Rigault, Aldering, Kowalski, et al. 2015, ApJ, 802, 20 
\end{thebibliography}}
\end{block}

\vskip1ex

\begin{block}{Acknowledgments} 
%using small, but could also use footnotesize, but other sizes are too small.
\small{
% {\bf NEED TO ADD GSU \& GPS FUNDING!} \\
Funding in part by NASA grant HST-GO-129, the Notebaert Professional Development Fund, and ND Graduate Student Union CPG. This research made use of Astropy, a community-developed core Python package for Astronomy (Astropy Collaboration, 2013).}
% \vskip1ex
\begin{center}
% \begin{tabular}{cc}
%     \vspace{53pt} 
%     \includegraphics[width = 4.5in]{ND_mark_black.pdf} &%$~~~~$&
%     % \vspace{-210pt}
%     \includegraphics[width = 4.5in]{astropy_powered.png}
% \end{tabular}

\includegraphics[width = 5.5in]{ND_mark_black.pdf}
\\ \vskip1ex
$~~~$\includegraphics[width = 5.5in]{astropy_powered.png}

\end{center}
\end{block}
\end{column}

\begin{column}{\sepwid}\end{column}

\end{columns}
\end{frame}
\end{document}