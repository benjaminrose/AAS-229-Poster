%%%%%%%%%%%%%%%%%%%%%%%%%%%%%%%%%%%%%%
% LaTeX poster template
% Created by Nathaniel Johnston
% August 2009
% http://www.nathanieljohnston.com/2009/08/latex-poster-template/
%%%%%%%%%%%%%%%%%%%%%%%%%%%%%%%%%%%%%%

\documentclass[final]{beamer}
\usepackage[scale=1.24]{beamerposter}
\usepackage{graphicx}           % allows us to import images
\usepackage{sidecap}
\usepackage{wrapfig}
\usepackage{caption}
\usepackage{sidecap}
\usepackage{cleveref} %add option [capitalize]?
\usepackage{subfig}
% \usepackage{natbib}


%-----------------------------------------------------------
% Define the column width and poster size
% To set effective sepwid, onecolwid and twocolwid values, first choose how many columns you want and how much separation you want between columns
% The separation I chose is 0.024 and I want 4 columns
% Then set onecolwid to be (1-(4+1)*0.024)/4 = 0.22
% Set twocolwid to be 2*onecolwid + sepwid = 0.464
%-----------------------------------------------------------

\newlength{\sepwid}
\newlength{\onecolwid}
\newlength{\twocolwid}
\newlength{\threecolwid}

\setlength{\paperwidth}{44in}%{48in}
\setlength{\paperheight}{33in}%{36in}
\setlength{\sepwid}{0.024\paperwidth}%should be 0.024
\setlength{\onecolwid}{0.22\paperwidth}%should be 0.22
\setlength{\twocolwid}{0.464\paperwidth}%should be 0.464
\setlength{\threecolwid}{0.708\paperwidth}
\setlength{\topmargin}{-0.5in}
\usetheme{confposter}
\usepackage{exscale}

%-----------------------------------------------------------
% The next part fixes a problem with figure numbering. Thanks Nishan!
% When including a figure in your poster, be sure that the commands are typed in the following order:
% \begin{figure}
% \includegraphics[...]{...}
% \caption{...}
% \end{figure}
% That is, put the \caption after the \includegraphics
%-----------------------------------------------------------

\usecaptiontemplate{
\small
\structure{\insertcaptionname~\insertcaptionnumber:}
\insertcaption}

%-----------------------------------------------------------
% Define colours (see beamerthemeconfposter.sty to change these colour definitions)
%-----------------------------------------------------------

\setbeamercolor{block title}{fg=dblue,bg=white}
\setbeamercolor{block body}{fg=black,bg=white}
\setbeamercolor{block alerted title}{fg=white,bg=dblue!70}
\setbeamercolor{block alerted body}{fg=black,bg=dblue!10}

\setbeamercolor{item}{fg=dblue}
\setbeamercolor{item projected}{fg=white,bg=dblue}

%-----------------------------------------------------------
% Name and authors of poster/paper/research
%-----------------------------------------------------------

\newcommand{\sn}{SNIa}

\title{Correlations Between Hubble Residuals and\\\vskip1ex Local Stellar Populations 
of Type Ia Supernovae} 
\author{Benjamin Rose \& Peter Garnavich}
\institute{Department of Physics, University of Notre Dame, Notre Dame, IN 46556\\ \vskip1ex brose3@nd.edu}

%-----------------------------------------------------------
% Start the poster itself
%-----------------------------------------------------------

\begin{document}
\begin{frame}[t]
\begin{columns}[t]

\begin{column}{\sepwid}\end{column}

\begin{column}{\onecolwid} \label{col1}
\begin{alertblock}{The Big Question}
% What information about Type Ia supernovae will allow us to reduce the scatter on the Hubble diagram?\\
% For Type Ia supernovae to be standard candles their variabilities need to be understood and corrected. 
% In order to improve distance measurements, how can Type Ia supernovae variabilities be better understood and the scatter on the Hubble diagram reduced?
Can using information on the local environment reduce Type Ia systematic
distance uncertainties?
\end{alertblock}

% \vskip1ex
%todo:emphaize luminosity-decline rate, color, $x_1$, host galaxy mass, global mentality, and local H$_\alpha$
% \begin{block}{Basic SNIa standardization}
% 
% \end{block}

\vskip1ex

\begin{block}{Environmental effects}
Type Ia supernovae (\sn{}) are excellent distance indicators, but their Hubble residuals show some correlation with host galaxy properties. 
% Type Ia supernovae (SNIa) are used as a standard candle to measure cosmological distances. 
Each individual \sn{} has some variability, but these can be standardized. The two most popular light curve fitters take into account relationships like luminosity-decline rate \cite{Phillips93} and \sn{} color.
Research over the past several years is indicating that external variables seem to also contribute to \sn{} variability. There are still uncertainties about how the host galaxy and the local environment influence the luminosity, color, and Hubble residuals of \sn{}. 
%These global trends have been researched including metallicity \cite{Hayden13}, and host mass \cite{Childress13}. Other?
%2013ge is one of three from Childress that year.
%Local effects have also been lookedAnd something else about local effect, $H_\alpha$ x2 \cite{Rigault13,Rigault14}. 
There has been research that show the effects of host galaxy mass \cite{Childress13}, global metallicity \cite{Hayden13}, and local H$_\alpha$ \cite{Rigault13}%Rigault14}.

% \vskip1ex
% \begin{itemize}
%     \item global metallicity \cite{Hayden13}
%     \item host galaxy mass (mass step) \cite{Childress13}
%     \item local H$_\alpha$ x2 \cite{Rigault13,Rigault14}
% \end{itemize}
% \vskip1ex
\end{block}

\vskip1ex

\begin{block}{Data collection}

\begin{itemize}
    \item \sn{} selection and parameters by Campbell (spectroscopic and photometrically) \cite{Campbell13}
    \item Local enviorment {\it ugriz} form Holtzman Scene Modeling Photometry (SMP) \cite{Holtzman08}
    \item Local stellar population model from Conroy's Flexible Stellar Population Synthesis (FSPS, \cite{FSPS1, FSPS2}) and Foreman-Mackey's pyFSPS \cite{pyFSPS}
\end{itemize}

\end{block}
\end{column}


\begin{column}{\sepwid}\end{column}\label{leftspace}


\begin{column}{\twocolwid}\label{col2}
\begin{block}{Analysis}
This analysis follows the work of Gupta \cite{Gupta11} but rather then using the \textit{ugriz} of the host from DR7 Stripe 82 co-added catalog we use the local values at the \sn{} from Holtzman.
\end{block}

\vskip1ex

\begin{columns}
\begin{column}{\onecolwid}
\begin{block}{Model Parameters}
% This talks about why I chose the model space I did. 

The local stellar population is modeled by a four parameter ``$\tau$-model'', where $\text{SFR} \propto e^{-t/\tau_{\text{SF}}}$. 
\vskip1ex

\begin{itemize}
    \item $\tau_{\text{SF}}$ -- the e-folding timescale of star formation
    \item $t_{\text{SF start}}$ -- when the star formation turns on
    \item $\log(Z/Z_{\odot}$) -- the metalicity of the local stellar population
    \item $\tau_{\text{dust}}$ -- the optical depth of the dust around starts older then $7~\text{Gyr}$ younger stars have 3 times the optical depth
\end{itemize}
% Where $\tau_{\text{SF}}$ is the e-folding timescale. This star formation turns on at $t_{\text{SF start}}$. 

\vskip1ex

% This model also contains changing metalicity and  talk about dust model.

Details about the values for each parameter are available in table, \Cref{tab:fsps-param}.
\end{block}

\vskip1ex

\begin{table}[]
    \centering
    \begin{tabular}{c c}
    \hline
    \hline
     FSPS Parameters    & Grid Values \\
    \hline
     $\tau_{\text{SF}}$ [Gyr] & 0.1, 0.5, 1, 2, 3, 4, 6, 8, 10 \\
     $t_{\text{SF start}}$ [Gyr] & 0, 1, 2, 3, 4, 5, 6, 7 \\
     $\log(Z/Z_{\odot}$)    & -0.88, -0.59, -0.39, -0.20, 0, 0.20 \\
     $\tau_{\text{dust}}$ & 0, 0.1, 0.3, 0.5, 1.0, 1.5 \\
    \hline
    \end{tabular}
    \caption{FSPS Model Grid Parameters}
    \label{tab:fsps-param}
\end{table}

Look we can create a diverse model space.

\end{column}

\begin{column}{\sepwid}\end{column}\label{centerspace}

\begin{column}{\onecolwid}

\begin{block}{Fitting Models}
Selecting "Best-fit"
Calculating Age.
\end{block}

\vskip1ex

\begin{block}{Local Properties\\Compared to HR}
Lorium Ipsum
\end{block}
\end{column}
\end{columns}

\vskip1ex

\begin{alertblock}{Why HST?}
Lorium Ipsum
\end{alertblock}
\end{column}


\begin{column}{\sepwid}\end{column}\label{rightspace}


\begin{column}{\onecolwid}\label{col3}
\begin{alertblock}{Why HST?}
Lorium Ipsum
\end{alertblock}

\vskip1ex

\begin{block}{What is next}
Lorium Ipsum
\end{block}

\vskip1ex

\begin{block}{References} 
% \bibliographystyle{apj}
%using small, but could also use footnotesize, but other sizes are too small.
\footnotesize{\begin{thebibliography}{99}
\bibitem{Phillips93}Phillips, M. M. 1993, ApJ, L105, 413
\bibitem{Childress13}Childress, Aldering, Antilogus, et al. 2013, ApJ, 770, 108
%I may want another 
\bibitem{Hayden13}Hayden, Gupta, Garnavich, et al. 2013, ApJ, 764, 191
\bibitem{Rigault13} Rigault, Copin, Aldering, et al. 2013, A\&A, 560, A66
\bibitem{Gupta11} Gupta, D'Andrea, Sako, et al. 2011. ApJ, 740, 92
\bibitem{Campbell13} Campbell, D'Andrea, Nichol, et al. 2013. ApJ, 763, 88
\bibitem{Holtzman08} Holtzman, Marriner, Kessler, et al. 2008, AJ, 136, 2306
\bibitem{FSPS1} Conroy, Gunn, \& White 2009, ApJ, 699, 486
\bibitem{FSPS2} Conroy \& Gunn 2010, ApJ, 712, 833
\bibitem{pyFSPS} Foreman-Mackey, Sick, \& Johnson 2014. \url{http://doi.org/10.5281/zenodo.12157}
% \bibitem{astorpy} Astropy Collaboration, 2013
% \bibitem{Rigault14} Rigault, Aldering, Kowalski, et al. 2015, ApJ, 802, 20 
\end{thebibliography}}
\end{block}

\vskip1ex

\begin{block}{Acknowledgments} 
%using small, but could also use footnotesize, but other sizes are too small.
\footnotesize{
% {\bf NEED TO ADD GSU \& GPS FUNDING!} \\
Funding in part by 
% NASA grant HST-GO-129, 
the Notebaert Professional Development Fund, and ND Graduate Student Union CPG.

\textit{Software:} Astropy, Numpy, Pandas, Matplotlib, Seaborn, FSPS, Python-FSPS
}
% \vskip1ex
\begin{center}
% \begin{tabular}{cc}
%     \vspace{53pt} 
%     \includegraphics[width = 4.5in]{ND_mark_black.pdf} &%$~~~~$&
%     % \vspace{-210pt}
%     \includegraphics[width = 4.5in]{astropy_powered.png}
% \end{tabular}

\includegraphics[width = 5.5in]{ND_mark_black.pdf}
\\ \vskip1ex
$~~~$\includegraphics[width = 5.5in]{astropy_powered.png}

\end{center}
\end{block}
\end{column}

\begin{column}{\sepwid}\end{column}

\end{columns}
\end{frame}
\end{document}